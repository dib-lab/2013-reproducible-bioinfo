\documentclass[ChapterTOCs,krantz2]{krantz} % Use krantz2 for 7" x 10" trim
\usepackage{url}
%size

% @CTB put acknowledgements at end
% @CTB check haddock foo
% @CTB Alexis, do you want to be last author?
% @CTB use latexdiff
% @CTB mention online fora??
% @CTB do we reference SWC arxiv paper?
% @CTB I feel like we maybe need to mention R in the languages...
% @CTB and RStudio in the IPython Notebook section, sigh

\begin{document}

\title{Dummy title} \author{Dummy author} \chapter*{Dummy chapter needed for
the \textbackslash chapterauthor command to work later}

\mainmatter

\chapterauthor{Likit Preeyanon*}{Microbiology \& Molecular
Genetics, Michigan State University, East Lansing, MI}
\chapterauthor{Alexis Black Pyrkosz*}{Avian Disease and Oncology Laboratory
(ADOL), East Lansing, MI}
\chapterauthor{C. Titus Brown}{Computer Science \& Engineering and
Microbiology \& Molecular Genetics, Michigan State University, East Lansing, MI}


\chapter{Reproducible Bioinformatics Research for Biologists}
\footnote{*These authors contributed equally.}

\section{Introduction}\label{intro}

\subsection{Computational analysis in pre-genomic era}

The birth of computational biology in the 1960s
and 1970s brought key bioinformatic discoveries, but many analyses
were conducted by examining entire datasets by eye and shuffling data by hand.
Protein sequences and alignments were distributed in the annual printed Dayhoff
atlases \cite{Strasser2010} that budding bioinformaticians 
would manually type onto punch cards.  In the 1980s,
the term ``bioinformatics'' was coined and the first major sequence databases
were established, necessitating more automated and scalable computational analyses.

In the 1990s, protein and nucleic acid sequences were becoming plentiful
(distributed on CD-ROM), scientific software packages and spreadsheet programs
were affordable, several phage and bacterial genomes were sequenced, and the Human Genome Project (HGP) was initiated with Sanger
sequencing.  In 2003, the pre-genomic era ended with the publication of the human genome\cite{Lander2001,Collins2003,HGP}.  The National Institutes of Health also
invested heavily in sequencing other organisms to aid in annotation of the 
human genome.  While complete genome
sequencing was possible, the sequencing methodology was expensive, laborious and time-consuming,
spurring scientists to develop the next generation of sequencing technology.
In the mid-2000s, faster and cheaper
sequencing techniques began began to emerge, resulting in an explosion
of data, with bioinformaticians racing to develop computational tools
to analyze and mine it\cite{Ansorge2009}.

\subsection{Computational analysis in the era of next-generation sequencing}

As sequencing becomes ever more affordable,
the grand genomic and transcriptomic datasets that were the
dream of many pre-genomic era biologists have become commonplace.  A single
experiment in a small research lab can inform on
thousands of genes or entire genomes, and small genomes can be sequenced and assembled in a few hours.
Initiatives such as ENCODE\cite{Encode2004,Myers2011}, 
1000 Genomes Project\cite{Altshuler2010}, 
Cancer Genome Project\cite{Cosmic2011}, Human Microbiome Project\cite{Lewis2012}, 
Eukaryotic Pathogen and Disease Vector Sequencing Project\cite{EPDVSP}, 
Clinical Sequencing Exploratory Research\cite{CSER}, 
Centers for Mendelian Genomics\cite{CMG}, Environmental Genome Project\cite{EGP}, 
and HapMap Project\cite{IHP} make vast amounts of data readily available for
download and analysis.
As the field comes closer to achieving the 
\$1000 genome\cite{Mardis2006}, waves of individual genomes will inundate the 
public databases, providing a rich information source for
researchers to analyze with a wide array of tools.  Further,
proteomics, metabolomics, medical imaging, environmental conditions, and many
other kinds of data are becoming readily available for integration.  As
scientists continue to push the edge of data analysis and integration, the integration of these
different data types is increasingly required.  The field has
advanced far from the eye/hand methods of the pre-genomic era and outstripped
the spreadsheets and single software packages of the early post-genomic era.  
Modern computational analyses are a major part of biological studies \cite{Zhulin2009}
and require analyzing gigabytes or terabytes of data in complex computational
pipelines, which typically involve running several distinct programs with custom
scripts porting data between them.  These pipelines start from quality control on raw data (or by downloading primary data from public databases) and pass the data through many steps of calculation, validation, statistics.  They end with
summarization and graphical vizualization to aid end-users in comprehending the
complex results.

Genomics has expanded the drivers of science from hypothesis (devise a question 
and design/conduct experiments in response) to include discovery (sifting through large datasets in search of patterns).  With this greater emphasis on 
on statistical analyses of large datasets and data-driven modeling, even wet-lab biologists are
increasingly finding themselves at the computer instead of the bench\cite{Merali2010}.  
However, many biologists lack a strong background in mathematics or
computer science\cite{Bialek2004,Pevzner2009}, 
and struggle to transition from a graphical
computer desktop environment to the command line interface required for
many analyses.  Further, while they
usually have been trained in good wet-lab practice,
they often have minimal experience with computational 
practice and lack the knowledge necessary to 
efficiently perform high-quality reproducible 
computational research\cite{Merali2010,Segal2007}.

\subsection{Concerns in Bioinformatic research}

% @CTB what about moving this entire section (down through ``Guidelines for
% getting started) up front, and putting the current introduction in as
% ``background''?

Biologists' lack of computational experience is a significant hindrance as
biology expands to include data- and model-driven methodologies.
While the obvious solution is to limit the computational aspect of
biology to trained bioinformaticians and computer scientists,
this is impractical for two reasons:
not enough skilled bioinformaticians are available (only a minuscule percentage
of US universities have bioinformatics undergraduate programs), and many
computer scientists are uninterested/uneducated in science or unwilling to pursue careers
that pay considerably less than a commercial position.  Moreover, the
% @CTB citation for pay/disinterest?
substantial background in biology required to make appropriate use of
data blocks computer scientists from quickly moving into bioinformatics.
As a result,
wet-lab biologists have begun to venture into computation in increasing
numbers\cite{Heaton2012,Kelly2008}.
They are usually long on data and short on time, so they
focus on learning the computational tools needed to analyze their specific
data, concentrating on rapidly processing data with the tools 
as opposed to understanding the tools' underlying
assumptions.  Even more troubling, there is a cultural and social gap
because many labs and programs do not consider bioinformatics
essential for their biologists\cite{Segal2007}.  A researcher seeking to analyze large datasets 
when few or none of his/her coworkers or superiors have
computational expertise may have no clue where to begin, and be given very little
time to find or develop appropriate tools.  This situation fosters an ad hoc
approach to bioinformatics.    

The effects of this lack of expertise can be dire:  
\begin{enumerate}
\item Many researchers download computational tools from the Internet 
or collaborators 
and use them on large datasets without first
running a known test set
(the computational equivalent of a control).  Many programs contain
technical or scientific errors that will be readily apparent when running 
test sets, but will be missed otherwise\cite{Kelly2008}. Errors will be
carried into downstream analyses, costing
hundreds of hours of compute and bench time, and 
potentially requiring retraction of papers when the errors are caught\cite{Merali2010}.  

\item Many tools only run on the command line, are difficult to install, lack
documentation, etc., and therefore software may be selected based on ease
of use, rather than accuracy and scientific relevance

\item With the trial and error
approach used to create custom pipelines, biologists can
lose track of which tools they ran, the order in which they ran them, and the
parameter sets used for each.  Many biologists have not carried 
the standard scientific practice of painstakingly recording
wet-lab procedures in laboratory notebooks
over into their computational research.

\item Many biologists use software with the default parameters.  
The defaults are frequently selected by the original
programmers to optimize processing of the original test data or were based on a
set of assumptions that was correct for the original study, but may not be appropriate
for the a different biologist's data or research question.  While some parameters 
are relatively insensitive such that the defaults are sufficient, others 
will produce wildly different results if varied slightly.  
A single parameter
can be the difference between one group's results being correct and another's
being wrong.  

\item Those biologists who program their own tools must decide how
to release and support their code.
Some labs post their software on a website but rarely update it.  
Frequently as soon as a programmer
leaves the lab, their code becomes unsupported and 
joins the online graveyard of dead and obsolete code.  
Some labs refuse to release their code\cite{Morin2012}, 
leading reviewers and collaborators to wonder if 
inaccuracies are being hidden.  
\end{enumerate}

The good news is that
all of these problems can be addressed using tools 
and practices that are already available.  

\subsection{Reproducible research is attainable in Bioinformatics using modern tools}

Many tools for reproducible computational research exist and are
already being used in Computer Science, Physics, and
Engineering.  These
tools are routinely used to quality control the data analysis process,
facilitate useful collaborations, and maintain laboriously developed programs
and pipelines in the long term.  While these tools may be new to many
biologists, they have been in production for many years and 
are well-tested with tutorials and online
documentation.  Investing time to learn the
tools and establishing good habits of using them yields a larger benefit:
errors are consistently detected and corrected early instead of being
discovered only after time-consuming downstream analyses and attempted wet-lab
verification (or after a paper has been submitted or published\cite{Miller2006}).  The more
computational methods employed by a laboratory, the more essential the tools
are to efficiency, correctness, and reproducibility.  Further, while 
use of these tools is currently optional in biology\cite{Ince2012}, researchers can 
expect that within a decade, most journals and granting agencies will 
require the appropriate use of tools and methods in computational biology
research; the National Science Foundation already requires detailed
discussion of data management.

\subsection{Guidelines for getting started} Our goal is to help those biologists who
have zero or little background in computation to
get started with good practices and tools for computational science. The following sections are structured to provide
introductory knowledge for those biologists venturing into bioinformatics and the
command line interface for the first time; intermediate knowledge for those
biologists ready to start programming; and advanced techniques for seasoned
programmers for improving programs and automating pipelines; and a related tools
section that names tools and concepts that readers can seek out once they
have established a basic foundation for reproducible research.

\textsl{Note: the following sections give overviews and simple examples of the tools, 
but readers are encouraged to use the resources listed at the end of
the chapter to find specific information and step-by-step tutorials.  Essentially, 
this chapter tells readers about existing tools and why they are important to use, 
but is not itself a course in computational research.}

\section{Beginner} Here we discuss simple practices that beginners can use
to establish a strong foundation for making their computational research
reproducible, emphasizing those that are practical for scientists who primarily
run other people's pipelines and are making the switch from a graphical user
interface to the command line. We describe basic practices such as working
on the command line and selecting a text editor.  

\subsection{Computing environment -- the new benchtop}

Just as much wet-lab biology work is done on a lab bench with routinely available tools
such as micropipets, shakers, and spectroscopes, computational work
is usually performed on a computer with routinely available data parsing and analysis tools.  In this section, we will discuss computing environments (also
sometimes called operating systems or platforms) and the general tools that
stand ready on the computational benchtop.  

\subsubsection{Unix/Linux operating system}

% @CTB check Unix -> UNIX...

Most biologists are aware of two primary computing platforms available:
Windows and UNIX systems (which include Linux and Mac OS X). 
In the United States, most people
learn basic computer skills on Windows machines.  
However, developers of bioinformatics
software primarily use UNIX systems because of the large existing ecosystem
of tools, most of which are open source -- this means anyone can freely use, modify, and
redistribute resources under open-source licenses.  The open-source or 
free-software community has long attracted programmers and other 
technically-oriented people who creatively solve problems.
The Free Software Foundation\cite{FSF} has numerous 
open-source collections of development
tools, libraries, licenses, and applications for the GNU/Linux system.
Consequently, biologists can get an operating system and all the bioinformatics
tools they need for free for all of their computers and clusters.  
Further, UNIX systems have traditionally been easier to use remotely than Windows
machines,
the machine is a desktop computer 
in another room, an institution's high performance compute cluster, or 
the cloud.

Making the switch from graphical user interface (GUI) based software 
to command line software is useful because
the power to remix and combine tools at the command line outstrips that
of most GUI-based software; as an analogy, GUI-based software is similar to a lab
that relies exclusively on commercial kits whereas the command line 
is like a lab that is also equipped with chemicals and instruments that can be used 
to supplement the kits or develop novel techniques.  Also, many
GUI-based software packages require users to manually click through an analysis,
while the command line can be used to write complete instructions for an
analysis and run many datasets simultaneously.   Automation ensures 
that each dataset is run according to the
same instructions (avoiding human error) and unshackles the biologist from
the computer.  
Some GUI and web-based programs are available 
for those biologists who want to build pipelines without using 
the command line (such as Taverna, Pipeline Pilot, and Galaxy).  
Further, most
cutting-edge bioinformatics tools lack a GUI,
partly because building GUIs is a time-consuming task that is
difficult to fund.  As a result of this, biologists need
basic command line navigation skills to use the latest bioinformatics software.  Biologists
who use Windows but wish to take advantage of this cache of software and tools
can download and install a free program such as Cygwin (or MSYS+Mingw32 or 
Microsoft Interix) to emulate a UNIX system
on their Windows machine, or can use PowerShell.

\textsl{Note: modern Macintosh machines are a
good compromise for modern research labs because they 
have a UNIX-based system with a friendly GUI for casual use.}

\subsubsection{UNIX tools -- the new benchtop tools}

Just as wet-lab biologists use simple tools like pipets, centrifuges,
incubators, and gel boxes individually and then combine them to perform a
specific procedure, so bioinformaticians use simple UNIX tools
that each perform a specific task and string them together into a pipeline.
Many UNIX tools come pre-installed in UNIX systems or are freely
available online. These tools are invaluable for the beginning
bioinformatician, particularly if that researcher has no programming experience, 
because they will perform tasks with a speed, customizability, and
reliability that custom scripts cannot easily match.  Here, we
introduce some of the most basic, useful UNIX tools for bioinformatics.

\paragraph{Shell}

The shell is a language as well as interpreter that reads and interprets commands
from a user. Any biologist who has opened a terminal or command line interface
and typed a command has used the shell. There are several types of shells available for UNIX
system, but bash is predominant. 
Bioinformaticians often use shell commands to
run tools and automate tasks such as running pipelines, backing up
data and submitting jobs on a computer cluster. While shell languages can be
used to develop full-fledged programs, this can be time-consuming 
because the shell is designed around operating system tasks rather than
data analysis.
Bioinformaticians usually use shell commands to perform routine tasks such as
sorting large datasets,
searching for specific data in a group of files, or sifting through a large log
file and printing only the data relevant to a given project; they write programs to perform 
more complex tasks.  Detailed below are
some of the most general and useful tools on
UNIX systems.

\paragraph{Grep, Sed, Cut, and Awk}

\emph{grep, sed, cut,} and \emph{awk} are tools for parsing text files.
\emph{grep} is used to
quickly search through a  text file for a given word or sequence
motif, similar to using a \emph{find all} command in a word processor.  \emph{sed} is
useful for replacing words or phrases,
similar to using the \emph{find and replace} command in a word processor.
\emph{cut} is
used for selecting a column of data in a text file.  \emph{awk}, among many other uses, 
can search through files containing many columns of data and only print
those lines or columns that are needed for a given application.  
Each tool is useful when the user needs to
perform a single task quickly, create a simple pipeline to accomplish a
combination of simple tasks, or process files that are too large to open in a
spreadsheet program.  \emph{grep, sed,} and \emph{awk} understand regular expression
syntax (covered in the Related section), which offers more robust pattern searching options.  

\paragraph{Apropos and Man}

% @CTB why the space at the beginning of the next line?
\ \emph{apropos}
is a program that
displays a list of programs related to a keyword. It is
useful for biologists who need to find a tool to perform a specific function without 
knowing the program's name. For example, if a biologist wanted to archive files,
he/she might use the \emph{apropos} command to search for an appropriate tool:\\* \\*
\texttt{\$ apropos archive}\\*

\noindent
\textsl{Note: \$ indicates the command prompt or where the user would begin typing.
The user would not actually type the \$.}

The output will vary depending on system, but should look like:\\* \\*
\texttt{jar(1)    - Java archive tool}\\*
\texttt{libtool(1)    - create libraries ranlib - add or update the table of
contents of archive libraries}\\*
\texttt{tar(1)    - manipulate tape archives}\\*
\texttt{unzip(1)    - list, test and extract compressed files in a
ZIP archive\ldots}\\*

\noindent
To learn more about each program, 
the biologist can look at the standard manual for each program using a
\emph{man} command (\emph{man} is short for manual).\\* \\*
\texttt{\$ man tar}\\*

\noindent
In this case, a man command will display a standard manual page, which
typically includes the name of the program, synopsis, detailed description, and
options as well as some examples. Here is an example of the first few lines of
the standard manual for the tar program.\\* \\*
\texttt{NAME}

\texttt{tar -- manipulate tape archives}\\* \\*
\texttt{SYNOPSIS}

\texttt{tar [bundled-flags <args>] [<file> | <pattern> \ldots] tar {-c}
[options] [files | directories] tar {-r | -u} -f archive-file [options] [files
| directories] tar {-t | -x} [options] [patterns]}\\* \\*
\texttt{DESCRIPTION}

\texttt{tar creates and manipulates streaming archive files.  This
implementation can extract from tar, pax, cpio, zip, jar, ar, and ISO 9660
cdrom images and can create tar, pax, cpio, ar, and shar archives\ldots}\\*

\noindent
\textsl{Note: Apropos and man commands serve as a reference, not a tutorial on how to use
a particular command. Biologists may need to search in Google,
Wikipedia, and other resources (such as \texttt{software-carpentry.org}) 
to find tutorials, examples, and other information.}

\paragraph{History and Script}

The shell automatically keeps a record of all commands used 
in a session.  Typing \emph{history} will print the list of commands.  
This tool is useful when a biologist is developing a computational procedure.  
Once the biologist has determined which commands and parameters 
are necessary to perform a required 
task, he/she can use the \emph{history} tool to view and save the commands for 
future use (see next subsection).

If a biologist needs to save an interactive session at the command-line, 
he/she can use the \emph{script} tool.  A record will be generated 
for all data output to the terminal window.

\subsubsection{Saving commands}

One of the advantages of the command line is
that biologists can save the exact commands and parameters used to perform a
computational procedure, as opposed to a GUI-based procedure where it is
difficult to record which buttons and options were used and the order in which
they were clicked.  At the most basic level, biologists can write the commands
in their bound lab notebook.  Another option is save the commands in a text
file as a rudimentary electronic notebook so the biologist can search for a
procedure later.   

Example of shell commands to be written/typed in a bound or electronic notebook:\\* \\*
\texttt{bowtie-build DataSet001.fa DataSet001.Index}\\*
\texttt{bowtie -m 1 DataSet001.Index DataSet001Reads.fq DataSet001.map }\\*

In the example, the biologist is using software that aligns short reads
to a reference sequence database.  To use the software, two commands are required:
(1) call the program bowtie-build to read the reference file
\emph{DataSet001.fa} and
return the output files with the prefix \emph{DataSet001.Index},
and (2) call the program bowtie with the parameter \texttt{-m 1} using the files from the
previous step and the read file \emph{DataSet001Reads.fq} as input and name the
resulting alignment file \emph{DataSet001.map}.  By recording these commands exactly
as typed in a notebook, the biologist will know the procedure used to generate
the alignment file.  If the biologist has more datasets, then these can be run
using the same commands, changing only the file names.  Further, when the
biologist is writing up the results several months later, he/she can include
the procedure so the results are credible and reproducible.

The most useful option is to create a short shell script.  While this may seem
daunting to a beginner, it is no more difficult than programming lab
equipment (e.g.\ creating a PCR program on a thermocycler), and just as the PCR
program is recorded in full in a notebook and used for all subsequent
experiments, so can the shell script be painstakingly written and then simply
used and referred to in later analyses.  


% @CTB what are these ` things? They show up incorrectly in the text, right?
Example bash script \emph{alignMyRnaSeqData.sh}\\* \\*
\texttt{\#! /bin/bash}\\*
\texttt{dataName=`DataSet001'}\\*
\texttt{params=`-m 1'}\\*
\texttt{bowtie-build \$dataName`.fa' \$dataName`.Index'}\\*
\texttt{bowtie \$params \$dataName`.Index' \$dataName`Reads.fq' \$dataName`.map' }\\*

In the example bash script, the biologist has converted the
previous example procedure into a series of commands with the name of the
dataset and parameter list turned into variables.  When the biologist needs
to run it, he/she will type \emph{bash alignMyRnaSeqData.sh} at the command prompt and
the instructions will be executed automatically.  The beginner can then
manually edit the dataset name or parameter list with a text editor (see next
subsection) each time it is run.  Users who
need to run the shell script on tens or thousands of files
will benefit from learning a few extra commands so they can make the script
loop through a list of file names.  Using the script ensures that each time a
dataset is run, the procedure remains the same, the output files are named
systematically, and the biologist saves time by not having to retype commands
or troubleshoot errors from typos.

Bash scripts are particularly useful when testing scientific software with
different parameter sets.  Biologists may not know that
unlike laboratory commercial kits that have been rigorously tested on a
variety of samples by experienced technicians, a significant
number of scientific programs are written by graduate students and other
academic researchers who are trying to solve a specific problem and optimize
the parameters to that particular system.  These default parameter sets are
usually untested with other types of data unless the software
has an active community of users who
have found many of the problems.  Therefore, biologists using new
software should start by running a dataset with a known result (a control) on
the default parameters, and then vary the parameters one by one to determine
their sensitivity.  (When possible, the parameters should be looked up in the
documentation and through Internet searches to determine whether they are set
to their optimal values for the current sample type.)  Bash scripts are useful
for testing parameter sets because biologists can set default parameters, run
the script, change one parameter, and run the script again, confident that the
only change to the procedure is the one they deliberately made.  In the
previous example, a biologist would vary the parameters listed in the params
variable for each run.  The results are more comparable and reproducible, and
automating the procedure speeds up the parameter optimization process.

\subsubsection{Text editors and IDEs -- the bioinformaticians' word processors}

Text editors are 
similar to word processors in that they are used to open,
create/modify, and save text files and source code, but different because they
do not save formatting characters or binary information in the file.
Therefore, they produce the clean, simple files that are needed for running programs
and analyzing data.  There are many freely available editors with features such
as graphical interfaces, programming language-specific syntax
highlighting, and advanced text parsing commands. 
We recommend that biologists learn cross-platform editors
(i.e.\ can be used on Windows, UNIX systems, etc.), particularly
those biologists who use Windows machines locally and UNIX systems
remotely.

% @CTB hmm, should Vi be capitalized?
Two editors are particularly popular among computational scientists: Emacs and Vi (or Vim).
Both editors are cross-platform and have features that can increase productivity
such as macros that can automate tasks, regular expressions to 
tailor/speed search and replace, etc.
Emacs and Vi can also be modified/customized using 
their internal scripting languages.
This feature makes these two editors highly expandable and flexible.
Numerous free plugins for both editors are developed and maintained by a
large user-based community, which also provides free support for new users.

Integrated development environments (IDEs) provide tool sets
including an editor, usually with advanced features, debugger, package manager,
and numerous plugins. The most popular IDEs such as 
Eclipse and Netbeans support many languages including
C++, Python, and Ruby, as well as HTML, PHP, and JavaScript for web
development. IDEs also provide a nice GUI, which is built on
top of command-line tools.  Moreover, 
online tutorials are freely available for users of all levels.  For biologists who 
intend to invest heavily in programming their own tools, IDEs may be a convenient step 
up from text editors.

\section{Intermediate} This section addresses programming, the keystone of
bioinformatics research.  Biologists will be introduced to 
programming languages (and how to select one), good programming practices for
developing less error-prone and more efficient code, program documentation for self and
general use, version control systems as electronic notebooks
and distribution methods, and controls for testing homegrown code.  These
sections are each intended as an overview of the tools and practices for
writing scripts that are shorter than a page or two in length.  References and
tutorials for learning the languages and tools are included at the end of the
chapter. 

\subsection{Programming} Programming is one of the most valuable skills for 
bioinformaticians. For example, just
as an experienced wet-lab biologist might quickly pour a gel to
purify a new sample, a bioinformatician will write a small
script to filter a raw data file. Some scripts only contain a few
lines of code, which are written for immediate use and then discarded.  
Therefore, most custom scripts are not documented, tested, or 
maintained.  Major problems arise when hastily-written scripts are blindly
reused for other projects or different datasets without proper quality control.  In
this section we discuss programming tools and approaches that are 
important for good computational lab practice.

\subsection{Programming languages}

A programming language is a set of human-readable instructions that is translated into
machine code to instruct a computer to perform a task.
They are generally categorized as \emph{interpreted} and \emph{compiled}.
Interpreted (scripting) languages (such as Python, Perl, and Ruby) use an
interpreter program to run programs in one step, 
when the user is executing the
program. Compiled languages (such as C, C++, and Java)
use a two-step approach; the first step, compilation, run 
during code development, produces an executable file that
contains only machine code. This file is then executed by the user. Both types of languages are widely used in
bioinformatics. While compiled languages are generally more difficult to learn
and time-consuming to use than their counterparts, the 
programs usually run faster because the code is translated in advance.
This makes compiled languages more suitable for processing large
datasets or performing complex or repetitive calculations. Interpreted languages are
easier to learn and use, especially for beginners; for processing 
small datasets or relatively simple
tasks, the difference in execution speed is negligible, while the advantage
in development time can be significant.

Biologists with no programming experience should consider using a widely-used 
interpreted language such as Python or Perl.
Both languages have a large user-based community, reference materials, and
high-quality third-party libraries.
While academia in particular has a culture of 
developing tools from scratch even when alternatives are available, this practice
is increasingly challenging because pipelines are now too complex for a single novice programmer 
to develop quickly and accurately\cite{Wilson2006}. Third-party libraries are developed 
by programmers other than the developers of the language itself, and extend the language
with custom written functions.  For example,   
biologists who want to develop a bioinformatics web application 
can use Django, a Python third-party library that already contains most of the
code required to build a Web application.  The use of libraries is encouraged because they
reduce errors in a program (i.e.\ a new programmer need not 
develop code that is susceptible to bugs when
polished, well-tested code is already available). It also reduces program size, 
which increases maintainability (the programmer's ability to fix 
bugs and update code as upgrades to software and scientific methods
become available).
Many libraries are actively maintained, developed, and used
by a community of programmers and scientists; therefore, they are well-tested
and fairly reliable. Many libraries also support users via tutorials, online
web forums, and mailing lists.


\subsection{Good programming practices}

Just as biologists follow standard lab practices in the wet lab, they
should follow standard computational practices when writing code.
For example, biologists frequently write short scripts to perform
simple tasks. While writing a script, the 
meaning of each command and program logic 
is easy to understand.  However, after a month or two, it can be
nontrivial to determine the code function, inputs, etc. that were once
so clear.
Furthermore, as scripts change over time, multiple versions of each script may be scattered 
over multiple computers. This problem is akin to a biologist performing a
quick procedure at the bench, writing scant notes, and not taking the time to a clean up or write down the procedure in detail.

In this section, we discuss basic programming
practices that help programmers organize their code and make their
scripts more reusable for other projects. We also introduce a type of
software, version control software, that facilitates
distribution of code among collaborators.


\subsubsection{Code documentation} 
All code should be documented in plain English.  
The main purpose is to inform users about code function, expected input
and output, and usage details, which are collectively called
a code description. This is
similar to laboratory equipment being packaged with standard operating
procedures and troubleshooting guides that are available to biologists
using the equipment.  The programmer should remember that just as the
wet-lab biologist will be more interested in using an instrument to do an
experiment than opening the control panel and tracing circuits, so will users
be focused on using a program to process data rather than reading
source code, and should write the documentation accordingly. 

Professional programmers conventionally write a code description at the
beginning of each short program.  Code descriptions should be short, 
providing maximum information in a
minimum of words.  For simple scripts, this may be two lines: one for the
description and one to describe the usage.  Some people might include the
name of the author, date created, and date modified in the code description;
however, when using a version control system (introduced later in this
section), this practice is redundant.  

When beginning programmers graduate to bundling scripts, programs, and libraries together,
they should add a README file, which is a text file containing
documentation for the entire code repository. The file usually includes
the name of the author, contributors,
installation, usage, and licenses for all programs in the repository.

Code readability is important.  
Experienced programmers customarily use two practices to improve readability:
descriptive variable/function names and concise but clear comments.  
For example, a new programmer might write the following code and comment 
(comment shown in italics):\\* \\*
\texttt{x += 1  \emph{\# add 1 to x}}\\

\noindent
Readers with minimal knowledge of Python will recognize that 1 is added
to a variable \emph{x}. The comment does not provide any information about the purpose
of the statement or describe \emph{x}. Instead, the code should be commented like this:\\* \\*
\texttt{x += 1  \emph{\# increase the count of DNA sequences read in}}\\

This way, a reader will quickly understand that \emph{x} is the number of DNA
sequences read and the number is being incremented by one.  A more useful way
to write this statement is:\\* \\*
\texttt{sequence\_count += 1 \emph{\# increase the number of DNA sequences read by one}}\\

\noindent
After choosing a more descriptive variable name, the comment is now largely
redundant and could even be omitted. 
This is a simple example of how code readability can be improved 
by choosing appropriate variable names and commenting code sparingly.

\subsubsection{Managing code/text with a version control system}

A common change tracking nightmare 
is when a programmer creates a script 
and sends a copy to a collaborator, who we'll call Adam. After a period of time, 
Adam finds a bug in the first version of the script and informs
the author. The author fixes the bug and sends Version 2
to Adam, not knowing that Adam
previously sent Version 1 to other collaborators, Beth and Celia, who are not privy to Version 2 
and therefore are likely to have unknowingly generated erroneous results with it.  
Further, Adam may inadvertently confuse Version 1
with Version 2 because the script name is the same for both and
the only difference is in the code, which Adam cannot or does not read. Later, 
Beth (with Version 1) might try to compare
results on a similar dataset with Adam's results from Version 2,
and spend considerable time tracking down the reason for the differences.
This would result in (at best) a re-discovery of the bug and (at
worst) attempts to publish the erroneous comparison in a journal.  The author, 
who has meanwhile upgraded to Version 5, 
is unaware of anything except that 
Adam was sent Version 2 a long time ago.  This convoluted 
comedy of errors is one reason why a technique known as version control is essential in collaborations.

A version control system
(VCS) is a program that tracks changes made to a file or set of files in a specified
directory and records them on a central server. Users can add, remove,
or edit files and the version control program will compare each file
with the previous version and record the differences. For a single programmer,
a VCS can be a rigorous, efficient electronic notebook of changes to a program:  As the programmer
creates/updates scripts, with a simple command, each change is meticulously
recorded, dated, and archived.  For a team of programmers, a VCS
is a group notebook, distribution tool, and collaboration aid.  Each programmer can access the
latest version of the code and make changes.  If multiple changes are made by
different programmers to the same file(s), the changes are noted and automatically
merged if possible. Conflicts (in which two
programmers modify the same line or region of code) are flagged for manual resolution.  
For nonprogrammers, a VCS is a
useful means of obtaining up-to-date software tools, as the VCS usually contains the most recently updated version of the software.  Users can download code
from the VCS and know exactly which version they are downloading.

\textsl{Note: VCSs are also commonly used when collaborators are writing 
a paper using LaTeX or a similar document markup language.  Each collaborator 
has immediate access to the most recent version of the paper, can make 
changes directly to the document, and commit them, as opposed 
to trading various versions via email or having one author decipher scribbled-on 
printouts.}

This is particularly useful when a research lab has one programmer 
and multiple users because the programmer can create/edit scripts in
the VCS and then users can check that VCS to determine if they need to
download new/updated tools.  When publishing 
computational analyses, study
authors can note the version numbers of the scripts used so biologists
wishing to reproduce the study can be given the correct version.
Also, when the programmer leaves the lab, the latest versions of the
tools are in the VCS so a newly hired replacement can immediately
access and start maintaining the code.

\textsl{Note: while setting up a VCS will require an initial investment of time, 
subsequent use is usually limited to a few simple commands.  Some VCSs
also include a GUI to help beginners and nonprogrammers use the system.}

Some major version control systems (Git, Mercurial, etc.) are associated with websites 
that host repositories for free.  For example, GitHub.com hosts more than 200,000 free code
repositories for open-source projects. These can be accessed
freely from any part of the world.  Using GitHub
simplifies distribution because individual research labs do not need to
constantly update lab websites with the latest version of a program;
they can point lab members, collaborators, and blog readers to the online
repository that the lab programmers are already updating.  It is becoming 
common to publish a link to a GitHub repository to
fulfill bioinformatic journals' requirement of open-access software, 
which is advantageous because the site is separate from
university or business websites that may change over time.  

\textsl{Note: biologists who prefer to keep their
code for internal use only can set up Git or Mercurial to work 
locally only, pay Github for private repositories, or use Bitbucket.  
Alternatively, they can set up their own secure
server (or have a network administrator set it up) and use
Git or Subversion (SVN).}

\paragraph{Real world example}

In our lab, every programmer has a Github.com account for their projects.
Some
projects on Github are also linked to the lab repository
\texttt{https://github.com/ged-lab}, which serves as the main repository for
all source code and other materials written by lab members. 
We include a link to the main repository in each
publication so anyone can download our source codes and materials
for using or reproducing results without first contacting us. 
In addition, anyone who finds a bug or wants to contribute to
the project can do so by simply cloning the project, editing or adding code
and submitting a request to merge a change to the project.
This opens up opportunities for improving the quality of scientific software
as well as collaboration.
This method has been proven quite successful in the open-source community.

\subsubsection{Basic code testing}

Roughly one to ten programming errors occur 
per thousand lines of code\cite{Boehm2005,Ince2012}.  
In this section, we discuss two techniques to help
programmers find obvious bugs upfront: 
assert statements and doctests. We introduce more advanced
tools such as unit tests and automatic testing systems in the \emph{Related}
section.

\begin{itemize}
\item Assert Statements

The purpose of an assert statement is to compare
a calculated value with an expected value and return true or false
based on a programmer-defined condition. Assert statements can be used to
test if code works as expected.  They are
particularly handy when testing \emph{edge cases} such as when a user
uses unexpected parameters or data files in unrecognized formats.
For example, if we write a function \emph{count\_gc} that returns the number of
G and C nucleotides in a sequence, we could use assert statements to test the
function:\\* \\*

\texttt{assert count\_gc("ATGTC") == 2}\\*
\texttt{assert count\_gc("ATTTTA") == 0}\\*
\texttt{assert count\_gc("") == None}\\*

The first line tests whether the \emph{count\_gc} function correctly counts the number
of G's and C's in the normal case or a mix of all four nucleotides.  The second
line tests if the function can correctly handle a calculation where there are
no G's or C's present, which is also expected to be a common case.  It is good
programming practice to always test cases where zero is the expected result to
ensure that it is correctly calculated and reported.  The third line tests if
the function recognizes that it has been passed an empty sequence and correctly
reports an error; \emph{None} is a defined value in Python that indicates ``no result''. In
Python, a programmer can also specify an error message if an assert statement
fails:\\* \\*
\texttt{assert count\_gc("") == None, "Empty sequence, should return None"}\\*

In this case, if the function does not return \emph{None}, the assert statement
fails and prints "Empty sequence, must return None" on the computer
screen.
This error message alerts the programmer that the \emph{count\_gc} function
is not handling the case correctly.  
% @CTB I would go with the correct answer being 0 here, FYI.  frac_gc might
% be a better example where a 0-length sequence should trigger an error. Thoughts?

Theoretically, assert statements should check all possible input
values; however, this is not usually practical. In the above example, it
would be impossible to generate every possible sequence that a user may
input to the function.  Therefore, a programmer will usually design
a representative set of input data to systematically test the code.  The
previous example demonstrated this by testing 
both common and uncommon cases.  The more assert statements
added, the more likely an existing error will be found.  

\textsl{Note: assert
statements are also useful 
when the programmer is modifying the code later.
Well-tested programs can more easily be modified and extended because the
tests ensure that changes that break the existing code are more likely to
be immediately discovered.}

\item Doctest

Doctest is a useful feature in Python and several other languages that helps the 
programmer document
and test his/her code simultaneously. This is particularly useful when writing
documentation for developers planning to use or extend functions, because
the doctests can ensure that the documentation examples are correct.
 Basically, doctest compares output
from the Python interpreter with user-defined output. The test fails if they do
not match. The doctest for the previous function would look like this:\\*\\*

\texttt{>>>count\_gc("ATGTC") \\2}\\*
\texttt{>>>count\_gc("ATTTTA") \\0}\\*
\texttt{>>>count\_gc("")\\None}\\*

\end{itemize}

% @CTB should we separate with some kind of ``discussion'' or ``testing'' headline?

Effective testing catches errors. However, the human programmer can almost
never consider and write tests for all possible ways of breaking code when
she is developing the first version of a program; bugs are not only inevitable
but unfortunately very common.
Therefore, writing tests should be incremental; each newly discovered bug
should prompt the addition of a new test.

Tests are useful on many levels, but some programmers still do not
write tests\cite{Wilson2006}.  One reason is that test writing is not formally taught in most
undergraduate computer science courses and therefore many programmers, let
alone biologists, lack the required knowledge or experience.  Another reason is
because it is time-consuming and not considered a critical path activity\cite{Segal2007,Kelly2008}.
Rigorous tests may contain more lines of code than the actual code. 
However, there is some evidence that
programmers who write tests 
spend less time debugging and produce higher quality code.
% @CTB cite ... crumb, who is it... one of the Wilson anthologies... @@CTB
Moreover, time spent repeating an analysis because of a bug is usually far
costlier than time spent writing tests.

For many biologists, the guidelines introduced in the Beginner and Intermediate
sections are sufficient to build a strong foundation for reproducible
computational research.  Beginners should be ready to investigate the tutorials
and resources listed at the end of the chapter to build knowledge and
experience with the command line interface: the new benchtop with open access,
high-quality UNIX tools for data processing (without programming), simple bash
scripts to generate reproducible procedures and optimize parameters, and text
editors/IDEs to create and manipulate files across platforms.  Intermediate
users
will be able to investigate various programming languages to find the one with
the most high-quality libraries and support for their research area, document
their code so it can be easily read and understood by other researchers, use a
version control system as an electronic notebook and up-to-date distribution
system for their evolving code, and write systematic tests to catch bugs early
in the development and analysis process. 

For most biologists, effectively using already developed software pipelines and writing
small scripts to port data between them or processing large results files with
UNIX tools is all they will need to complement their bench work.  By
incorporating these tools into their computational research and observing the
computational lab safety practices, biologists can work
effectively on the new benchtop, produce timely, accurate results that are
simple to repeat with bash scripts or short, well-tested programs, and can
distribute their programs per journal requirements using online version control
so the research community can spend less time reinventing and fixing code and
more time advancing science. 

\section{Advanced}

Once a biologist has built a strong foundation for reproducible computational
research, he/she may wish to progress to more complicated analyses, which
accordingly require more complex calculations.  The resulting programs can
contain hundreds or thousands of lines of code, more instructions than a single
human can keep in his/her head.  An eager biologist who has been developing 
scripts with less than 50 lines may jump in and create a single file of
several hundred lines.  However, once a program has advanced beyond the
simple script, new programming practices need to be followed to produce usable
and maintainable code.  This is analogous to chemists running a small reaction
in the lab vs.\ chemical engineers scaling up a reaction to run in a chemical
plant; process and resource management become significantly more important.  As in the
previous sections, these practices facilitate reproducibility,
productivity, and frequently help maintain the programmer's sanity.  In this
section, we will discuss modularity (the practice of writing code in small blocks), refactoring/optimizing code performance for use with
modern huge datasets, and using the IPython notebook as an interactive
notebook/computing environment for integrating different programs and platforms
and performing a complex analysis from start to finish.

Biologists/bioinformaticians grapple with a major problem before they start 
designing a new program:
absence of detailed program specifications\cite{Segal2007}.  Ideally, industrial
programmers have a complete set of specs from the customer 
and can design an elegant software solution before they begin
coding.  In the bioinformatics lab, programmers frequently have only a piece of
the problem laid before them and minimal input from lab members.  
Bioinformaticians need to quickly
develop a program that reads in data, performs a calculation, and writes
results so lab members will react to the results, try to validate them, mention
specifications/expectations that were not stated initially, or
rethink the problem.  The programmer is then
expected to refine the program and show new results to
solicit more feedback iteratively. 

Productivity is the key to using this evolutionary approach to problem solving.  
Writing robust, reusable, and maintainable code is traded for
writing code quickly because the programmer assumes that most code will be
modified or discarded during the development process.  Therefore,
biologists/bioinformaticians should write code that is 1) functional, 2)
readable, and 3) testable.  Functionality is most important because if code
does not work, then it is considered worthless\cite{Kelly2008}.
Readability is necessary so the code can be understood by all programmers and users
on the project.  
Testability is required
because larger programs have more ways to break and therefore even more tests
are needed.

\subsection{Modularity}

The first good practice for writing high-quality large programs is to divide
code into small modules.  A module is generally a small block of code that
performs one specific task such as reading FASTA files, ensuring that a DNA
sequence contains only the characters A, C, G, and T, or calculating the
average of a set of numbers.  The short scripts produced by intermediate
biologists/bioinformaticians can easily be converted into modules; experienced
software developers frequently write scripts that are simultaneously both.  
The purpose of creating modules is to take
advantage of all the tools and practices discussed in the Intermediate section.  
It is also easier for a programmer to write logical and organized
code when creating several small modules and linking them together than when
writing one long linear program.  These modules can then be bundled together to
form a programmer's custom library.  A large program ideally should consist of
a main program file that accepts user input and then passes it through a series
of modules or library functions.  

There are several additional advantages to this practice:  
\begin{enumerate}

\item Simplify and speed up programming.  For example, a program may need to read
several FASTA files to function.  In a linear program, the
code to load a file would need to be copied and pasted several times,
which decreases the readability of the code.  If the program is modular, the
programmer need only create one module in the library and then reference it as
many times as needed in the main program.  

\item Library modules are easier to maintain.  
In a long, linear program that reads multiple files, if a bug is found when
reading the first FASTA file, then the error is likely to be in all instances
of the code.  Novice programmers frequently only fix those
instances where an obvious error is shown, leaving silent errors in other parts
of the program.  In the modular example, the bug is fixed once in the module.  
Similarly, the module need only be
tested once whereas good testing of a linear program would require many more
tests.

% @CTB not a big fan ``linear'', but not sure how to convey otherwise. I'll
% think on't.

\item Modules are
easier to reuse for other projects.  Once a programmer has written and tested a
module and added it to his/her custom library, it
can be used for all future projects.  If the program is linear, the programmer
must copy and paste lines of code from an old program to a new one, and then
diligently test the code, or the
programmer will need to reinvent the wheel by writing fresh code to load FASTA
files for each new program.  

\item The programmer can easily combine custom
modules and third-party libraries to quickly and efficiently create large
programs.  

\item A team of programmers can easily collaborate on a large
program when each of them is writing/testing different modules.  

\end{enumerate}


\subsection{Code refactoring}

Refactoring is the process of changing a program so the code is different but
the results are same.  This is similar to changing a wet-lab procedure  
so it uses fewer consumables, less dangerous chemicals, and less time while still
generating a result that is of similar quality.  Once a programmer has a
well-tested program that produces the desired results, he/she can refactor the
code so it is more readable and simpler, with emphasis on the readability.
Before beginning, test codes should be written to ensure that the code still
runs properly after refactoring and the code itself should be self-documenting
(see Intermediate section). 

When refactoring, there are several preferred practices:  
\begin{enumerate}

\item Remove programming
language-specific idioms or overly complex statements to increase readability.
Many programming languages have similar structures and syntaxes (i.e.\ a Perl
statement can be read by a C++ programmer as long as it does not
use Perl-specific syntax).  Also, overly complex statements 
tend to contain errors.  

\item Divide large
functions or modules into smaller ones.  As discussed in the
modularity section, smaller modules are easier to code, test, and reuse.  

\item Remove dead/obsolete code.  Because of the evolutionary process 
discussed previously,
nonfunctional code from earlier versions might be lurking in the code.
Removing obsolete lines will both increase readability and ensure that compute
resources are not wasted on useless processing.
\end{enumerate}

\subsection{Code optimization}

As mentioned previously, interpreted
languages are often used to develop bioinformatics software because they
reduce development time with the acceptable tradeoff that they are
slower than compiled languages. In some cases, however, performance is critical
and the program may be considered useless if it cannot achieve a particular
speed.  
Optimization involves identifying the bottlenecks (the slowest
sections) and modifying them to use different algorithms or
compiled languages.  This is similar to a wet-lab biochemist taking a multistep
synthetic pathway, determining the rate-limiting steps, and either
substituting those steps with new reactions or using catalysts.  Generally, 
only the
bottlenecks should be optimized; opening one or a few bottlenecks is usually
sufficient to achieve the desired performance without spending time optimizing
the entire program.  The bottlenecks can be found using a tool called a profiler,
which reports the time and the number of times a particular
function or a method is called.
Examples of a profiler are GNU gprof for C/C++ and profile,
cProfile, and pstats module for Python.

\textsl{Note: some common bottlenecks in bioinformatics programs involve 
loading data multiple times, unnecessary reading/writing data to disk, 
and inefficient searching using nested loops.  These can often be solved 
by using more efficient algorithms or data structures.  
Novice programmers are prone to writing algorithms that 
mimic how a human would perform a task instead of 
harnessing the abilities of the computer 
and the specific programming language.  Consultation with other bioinformaticians or with an online forum is a good path to improving the efficiency of your code.}

Optimization depends on the
languages being used. For a bioinformatics program written in Perl or Python, a
bottleneck function can be rewritten in a compiled language like C++ or
Fortran, and then wrapped so the interpreted language can use it.  This method can increase performance by several magnitudes,
although poorly written code in a compiled language may not function as well as
well-written code in a scripting language.  Experienced
biologists/bioinformaticians utilize data structures and libraries that are
built into the scripting and compiled languages, which have been optimized by
professional software developers.  For example, in Python, many built-in data
structures and functions are actually implemented in C and wrapped so
they can be called using Python code.  A biologist/bioinformatician need only 
find an appropriate method that has already
been optimized and implemented in a compiled language.  

\subsection{Research documentation}

As stated previously, a complete bioinformatic analysis usually
consists of running several third-party software packages, scripts that port
data between them, and visualization tools to represent the final
results.  To reproduce the results, a biologist must
repeat every step in the analysis in the correct order with the appropriate
parameters. Until recently, providing a
complete set of instructions was not trivial.
Because the biologist/bioinformatician is frequently using an evolving
procedure, the bookkeeping required for recording detailed procedures
can become complicated.
Fortunately, scientists and companies have developed tools to simplify
the process, allowing researchers to conduct computational analyses while
simultaneously building the final set of instructions. 
In this section, we introduce the IPython notebook, which has become
popular due to its support of Python, shell commands, and R (a
statistics and graphing language). 
 
\subsubsection{IPython notebook}

IPython notebook is a combined
electronic notebook and programming/computing environment.  
Users can create a notebook for a project and link all scripts,
programs, and shell commands, and parameters used to the notebook, including
the order in which they are to be run.  Users can then run the analysis from start to
finish in the notebook and view/save output at each step as well as add textual
notes and comments.   This high level of organization can boost
biologists/bioinformaticians' productivity while giving them the tools to run
and rerun their analyses reproducibly.  A single click will run an entire
pipeline, expediting parameter optimization, replicate runs, and reruns after
fixing bugs.  Automating the process in the notebook minimizes mistakes from
typos and other human errors.  Moreover, IPython notebook can be run on a
remote server, making it suitable for computer clusters or
cloud computing systems. The notebook can be distributed to
collaborators who can rerun all commands and see identical results on their
computer without the programmer writing any additional documentation.

\textsl{The notebook is not efficient
for running processes that require days to finish.  Therefore, we write
shell scripts to perform the laborious number-crunching and use the notebook
for everything else.}

The IPython notebook is built on top of IPython, an advanced Python shell for
interactive Python programming, but can support many more programming
languages.
For example, with the rmagic plugin and Python RPy2 library, users can
use R libraries such as those from Bioconductor (see \emph{Resources})
and store the results in Python data structures, which can be further analyzed 
with Python.
This feature reducing the number of steps in
data transformation, which is a common problem in
bioinformatics research.
Another useful feature of IPython notebook is an in-line plot, which allows
users to use Matplotlib library or R to make and visualize a plot, and then save it
in IPython notebook with comments and code to create an executable document.
Finally, the notebook is stored as a plain text file that can be version controlled
and distributed as discussed above.
IPython has many more plugins and shortcuts that support scientific
computing analyses.
Shown below is an example of boosting productivity by using a shortcut:\\*\\*

\texttt{>>> expression\_values =!cut -f 2 expression.dat \emph{\# read in a
value from the second column of a text file}}\\*

Expression values are contained in expression.dat, and 
\emph{cut} is used to select the second data
column (\emph{-f 2}).  This data is then assigned to the Python list variable
\emph{expression\_values} for later use.  Running this single line in
the notebook can take the place of running the \emph{cut} command at the command line,
saving the results to a file, opening that file in Python, reading the data,
and then assigning it to the list variable.  Throughout this chapter we have
emphasized the idea that biologists/bioinformaticians can benefit greatly by
utilizing existing well-tested tools as opposed to reinventing the wheel.
IPython has been used by many scientists for several years; therefore, it is
not surprising that there are many commands, shortcuts, and plugins that
perform common tasks elegantly, accurately, and expeditiously.  
IPython also allows users to
create their own plugins to extend its functionality.

\paragraph{Real World Example}

% @CTB do you mean first author of THIS manuscript or THAT manuscript?
In a recent manuscript, the first author (C. Titus Brown, MSU) completed parts
of the analysis in the IPython notebook, available at
\texttt{https://github.com/ged-lab/2012-paper-diginorm.git}. The author also
provides a tutorial on running the pipeline and reproducing the
results using IPython notebook at
\texttt{http://ged.msu.edu/angus/diginorm-2012/pipeline-notes.html}. 
The analysis was tested on Amazon cloud service with
\emph{ami-61885608}, which has all the required programs pre-installed. Anyone
can follow the pipeline and use the notebook to reproduce the
analysis from start to finish with identical results with minimal effort.

At the NIH Cloud Computing for the Microbiome workshop in 2012, a 
team of researchers from different backgrounds used IPython notebook 
and Amazon EC2 as a collaboration to quickly produce publishable results time\cite{RaganKelley2012}.

\section{Related Topics}

For biologists/bioinformaticians who have progressed through the Beginner
to Advanced sections, we briefly describe more advanced computing topics 
that can facilitate accurate, efficient computational research.

\subsection{Advanced tools}

%\subsubsection{Text editor macros}

\subsubsection{Regular expressions}

Regular expressions are tools for
searching text for a particular pattern of letters/numbers/symbols.  For
example, a bioinformatician can search for short DNA sequence motifs in 
a data file with an unstructured or unusual format.  Regular expressions have their own
syntax for defining a specific pattern, which to the casual eye can look like
an unintuitive shell language. For example, \texttt{logy\textbackslash{}b} defines a pattern for a word
that ends with \emph{logy}, matching biology, physiology, technology, etc.
In programming, regular expressions are
used to concisely and quickly search for patterns in data.

\subsubsection{Debugger}

Code frequently contains errors or bugs that cause unexpected results.
While syntax errors will be caught by the interpreter or compiler, 
logic errors will not be detected.
The debugger tool allows programmers to interact with their program such as
running, pausing, stepping in and out of functions or loops, and changing values
in variables during execution to locate a bug.
Most programming languages have at least one debugger.
GDB is a standard GNU/Linux debugger that can be used with most
compiled programming languages.
Moreover, major IDEs such as Eclipse and Netbeans have built-in debuggers
with GUIs. 

\subsubsection{Unit tests and automated testing}

Unit testing consists of writing
code that tests individual units of a program. 
Each subunit is tested in isolation; therefore, tests 
on a given subunit will not be affected by bugs from other subunits. This
procedure helps locate errors in a large program. 
Unit testing libraries are available for most major languages
and help users create test
suites. Some libraries also
provide a test runner to run tests automatically. An important
advantage of automatic testing is that users can
test the program to ensure that the installation process is bug-free.  Unit
testing also promotes a test-driven development process, which 
helps guarantee that
every function works as expected and tests are written for every function in a
program.

\paragraph{Real world example}

In our lab,
large programming projects have separate folders for test code. However, each project uses
different libraries for testing; for example, Gimme
(\texttt{https://github.com/likit/gimme}) uses a Python unittest module whereas khmer
(\texttt{http://github.com/ged-lab/khmer.git}) uses user-defined functions, which are
recognized and run by the nosetest module. Instructions for automatically
running tests are included in the corresponding README
files.

\subsection{Advanced programming topics}

\subsubsection{Object-oriented programming paradigm}

Object-Oriented Programming (OOP) is a standard programming paradigm that
creates "objects," which usually consist of data 
and specific methods for operating on that data. This is useful when a large program 
needs to have standardized methods for data-processing but also uses 
several different types of data.
In addition, objects can be reused or extended without rewriting them from scratch.
This practice has been extensively used throughout the software industry and 
in scientific programming because it promotes code maintenance and expansion.

\subsubsection{Algorithms and data structures}

Most major languages used in
scientific computing provide libraries supporting well-optimized algorithms and
data structures. However, using a
pre-implemented algorithm without understanding the underlying concepts is
unsafe. Most algorithms have strengths and weaknesses that should be 
evaluated based on the specific application. Basic knowledge of algorithms 
will help biologists make a correct decision.

\subsubsection{Compiled languages}

While time-consuming, learning a compiled language such as C/C++
gives insight into how a computer functions because it requires 
machine-based knowledge. Scripting languages are designed to abstract 
away many low-level
details to improve programmer productivity. As a result, it also abstracts away
important concepts of computing such as memory management. A basic
understanding of how to program in a compiled language will help biologists
write better code using scripting languages because of a greater understanding
of the underlying mechanisms. In addition, many bioinformatics software
packages still use code in compiled languages to do rapid
processing and precise floating point calculations.  

\section{Conclusion}\label{conclusion}

The general problem in the
bioinformatics field is not an absence of tools and good practices, but rather
that many researchers lack knowledge and training with them\cite{Merali2010}.  Embedded in a scientific
culture that is relatively inexperienced about good computational practices, 
many biologists make their
first foray into bioinformatics with only intuition and the Internet to guide
them.  The tools and practices discussed here are intended to help
those biologists build a solid foundation in reproducible computational
research.  Because it would be impossible to condense several years of
scientific computing and data analysis training into a single book chapter, we
have focused on describing those tools and practices that are particularly
useful for novice bioinformaticians, emphasizing their contribution to 
productivity and reproducibility, with the intention of giving biologists the
introduction they need to then seek specific information and step-by-step
tutorials elsewhere. 

Investing in learning computational tools and practices will yield incalculable
return over the course of a biologist's career.  With the enormous potential
for discovery available in this era of Big Data, datasets can be expected to
continue expanding.  A biologist who starts learning and applying general tools
and good computing practices now will eventually save \emph{years} of time.
Moreover, the biologist who invests time in writing tests, using version
control, and automating/distributing analyses in the IPython notebook will both
be compliant with journal and granting agency policies for distributing code
and avoid the aggravation of other scientists repudiating his/her
unreproducible results.

While the list of tools and practices to learn might seem overwhelming to
biologists with no prior computational experience, we encourage them to take a
systematic approach to the education process.  Just as a wet-lab researcher is
trained on one instrument at a time, practices using it for
his/her current project, and then moves on to more complicated methods, so too
can he/she learn computational techniques.  Installing software for
running a command line interface and looking through a text file of data is a
good start.  This simple task can build confidence with using the command line,
and soon the biologist will be ready to learn simple \emph{grep} commands to make
looking through that text file easier and faster.  Each skill will build upon
the last, and the biologist will soon be applying these skills to his/her
research, finding experiments where a new option or tool will
accurately process data in seconds that would otherwise have required hours of
mind-numbing clicking.  Once the biologist gains experience with the Beginner
tools and practices, he/she will be ready to tackle the Intermediate section,
and the tools and practices described will seem like a natural progression.  As
the biologist becomes a more practiced bioinformatician with a well-stocked
toolbox, so too will his/her research advance.  

A single biologist using good computational tools and
practices can produce a lifetime of biological breakthroughs and innovation.  A
team of skilled biologists can work in parallel to push the
limits of biological knowledge in a particular area by several lifetimes.  As
more biologists practice reproducible computational research, the sheer breadth
and depth of their work will collectively move the entire field of biology into
a new era of scientific discovery.

\section{Available Resources}
\subsection{Books}
\subsubsection{UNIX/Linux tools}
\begin{itemize}
\item Haddock \& Dunn \emph{Practical Computing for Biologists}
\item Newham, Cameron \emph{Learning the bash Shell} [O'Reilly]
\item Robbins, Arnold and Dougherty, Dale \emph{sed \& awk} [O'Reilly]
\item Cameron, Debra et al. \emph{Learning GNU Emacs} [O'Reilly]
\item Robbins, Arnold et al. \emph{Learning the vi and Vim Editors} [O'Reilly]
\item Neil, Drew \emph{Practical Vim: Edit Text at the Speed of Thought} [Pragmatic Bookshelf]
\item Chacon, Scott \emph{Pro Git} [Apress]
\end{itemize}

\subsubsection{Python}
\begin{itemize}
\item Campbell, Gries, Montojo and Wilson \emph{An introduction to computer
science using Python} [Pragmatic Bookshelf]
\item Lutz, Mark \emph{Learning Python} [O'Reilly]
\item Model, L Mitchell \emph{Bioinformatics Programming Using Python} [O'Reilly]
\item Vaingast, Shai \emph{Beginning Python Visualization} [Apress]
\item Arbuckle, Daniel  \emph{Python Testing: Beginner's Guide} [Packtpub]
\end{itemize}

\subsubsection{Others}
\begin{itemize}
\item Joe Pitt-Francis, Jonathan Whiteley \emph{Guide to Scientific Computing in C++} [Springer]
\item James Tisdall \emph{Beginning Perl for Bioinformatics} [O'Reilly]
\item James Tisdall \emph{Mastering Perl for Bioinformatics} [O'Reilly]
\item Ellie Quigley \emph{Perl by Example} [Prentice Hall]
\item Peter Cooper \emph{Beginning Ruby} [Apress]
\item Joseph Adler \emph{R in a Nutshell} [O'Reilly]
\item Peter Dalgaard \emph{Introductory Statistics with R} [Springer]
\item Paul Teetor \emph{R Cookbook} [O'Reilly]
\end{itemize}

\subsection{Online resources}
\subsubsection{UNIX/Linux Tools}
\begin{itemize}
\item GNU Operating System\\\texttt{http://www.gnu.org}
\item Cygwin (Linux emulator for Windows)\\\texttt{http://www.cygwin.com}
\item MSYS+MinGW \\textt{http://www.mingw.org/wiki/MSYS}

\item Vi and Vim\\ \texttt{http://www.vim.org/index.php}
\item Emacs\\ \texttt{http://www.gnu.org/software/emacs/}
\item Github: Git online repository\\ \texttt{http://github.com}
\item Git tutorial \\\texttt{http://git-scm.com}
\item Mercurial \\\texttt{http://mercurial.selenic.com/}
\item SVN \\\texttt{http://subversion.apache.org}
\end{itemize}

\subsubsection{Python}
\begin{itemize}
\item Python: Python official website\\ \texttt{http://python.org}
\item Python style guide\\ \texttt{http://www.python.org/dev/peps/pep-0008/}
\item The Zen of Python: A guideline for Python coding \\\texttt{http://www.python.org/dev/peps/pep-0020/}
\item Learn programming by visualizing code execution \\\texttt{http://www.pythontutor.com/}
\item Python doctests \\\texttt{http://docs.python.org/library/doctest.html}
\item Python unittest \\\texttt{http://docs.python.org/library/unittest.html}
\item IPython: Advanced Python shell \\\texttt{http://ipython.org}
\item Scipy: Scientific tools for Python \\\texttt{http://www.scipy.org/}
\item Matplotlib: Python plotting library \\\texttt{http://matplotlib.org/}
\item Python: Speed and Performance tips \\\texttt{http://wiki.python.org/moin/PythonSpeed/PerformanceTips}
\item Performance analysis of Python programs \\\texttt{http://www.doughellmann.com/PyMOTW/profile/}
\end{itemize}

\subsubsection{R}
\begin{itemize}
\item R Official website: \\\texttt{http://www.r-project.org/}
\item Rseek: Search engine for R related materials \\\texttt{http://rseek.org}
\item Bioconductors: R packages for bioinformatics \\\texttt{http://bioconductor.org}
\end{itemize}

\subsubsection{Web Forums}
\begin{itemize}
\item BioStars: Bioinformatics answers \\\texttt{http://www.biostars.org/}
\item Stack Overflow: General programming\\ \texttt{http://stackoverflow.com/}
\end{itemize}

\subsubsection{Others}
\begin{itemize}
\item Software carpentry: Online training
\\\texttt{http://software-carpentry.org}
\item Rosalind: Learning Bioinformatics \\\texttt{http://rosalind.info/problems/as-table/}
\item Reproducible Research \\\texttt{http://reproducibleresearch.net}
\item Analyzing Next-Generation Sequencing Data\\\texttt{http://bioinformatics.msu.edu/ngs-summer-course-2012}
\end{itemize}

\bibliographystyle{plain} \bibliography{rr-refs}

\end{document}
